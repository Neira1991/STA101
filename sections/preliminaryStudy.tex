% !TEX root = preliminaryStudy.tex

\subsection{Nettoyage des données}
Nous avons 309 individues sans city don on va supprimer cette variable.

\subsection{Description des données}
Ci-dessous se trouve une description de chacune des variables de l'ensemble de données, à l'exception de la variable \textit{fullVisitorId} lequelle est un hash composé de 16 caractères alphanumériques pour identifier le visiteur.



\begin{center}
\begin{longtable}{ |p{3cm}||p{3cm}|p{2cm}|p{5cm}|  }
 \hline
 \multicolumn{4}{|c|}{Variables} \\
 \hline
 Nom & Type &Intervalle&Définition\\
 \hline
 hits   & Quantitative discrète  &[9,800]&  Les hits sont les interactions des utilisateurs sur le site web.\\
New visit &   Binaire   & [0, 1]   & C'est la première visite sur le site.\\
Page views&   Quantitative discrète  & [1, 137]   &Est une page chargée (ou rechargée) dans un navigateur.\\
Total transaction revenue & Quantitative continue  &[8.950, 871.380]&  Le revenu associé à la transaction. Cette valeur peut inclure les frais d'expédition, les taxes ou d'autres ajustements du revenu.\\
Transactions    &Quantitative discrète  & [1, 6] &  Nombre de transactions du client.\\
Time on site&   Quantitative continue  & [68, 8558]& Temps sur le site avant d'effectuer la transaction \\
 Browser name& Qualitative nominal  & -   & Nom du navigateur web.\\
 Browser width& Quantitative continue  & [320, 2560]& La largeur de la fenêtre du navigateur \\
 Browser height& Quantitative continue  & [280, 1356]& La hauteur de la fenêtre du navigateur\\
 Device category& Qualitative nominal   & -&Le dispositif par lequel l'utilisateur a effectué la transaction.\\
 Operating system& Qualitative nominal  & -&Le système d'exploitation par lequel l'utilisateur a effectué la transaction\\
 Language& Qualitative nominal  & -& La langue du navigateur au moment de la transaction.\\
 Country& Qualitative nominal   & -& Le pays depuis où l'utilisateur a effectué la transaction.\\
 Payment method& Qualitative nominal   & -&  Représente  ce que l'utilisateur a utilisé pour payer la transaction.\\
 Item count& Quantitative discrète  & [1, 18]& Un item représente un produit individuel acheté dans le cadre de la transaction. \\
 \hline
\end{longtable}
\end{center}

\subsection{Description des variables}

\subsection{Études des variables quantitatives}

\begin{table}[!htbp] \centering 
  \caption{} 
  \label{} 
\begin{tabular}{@{\extracolsep{5pt}}lccccccc} 
\\[-1.8ex]\hline 
\hline \\[-1.8ex] 
Statistic & \multicolumn{1}{c}{Mean} & \multicolumn{1}{c}{St. Dev.} & \multicolumn{1}{c}{Min} & \multicolumn{1}{c}{Max} \\ 
\hline \\[-1.8ex] 
hits & 86.618 & 99.984 & 9 & 800 \\ 
newVisits & 0.692 & 0.462 & 0  & 1 \\ 
pageviews & 16.313 & 14.863 & 1  & 137 \\ 
totalTransactionRevenue & 92,534.270 & 81,442.280 & 8,950  & 871,380 \\ 
transactions & 1.107 & 0.483 & 1 & 6 \\ 
timeOnSite & 888.189 & 959.672 & 0  & 7,547 \\ 
browser\_width & 1,146.060 & 653.819 & 320 & 2,560 \\ 
browser\_height & 764.110 & 156.292 & 280  & 1,356 \\ 
itemCount & 4.019 & 3.252 & 1  & 18 \\ 
\hline \\[-1.8ex] 
\end{tabular} 
\end{table} 


\subsection{Études des variables qualitatives}

Etudes
