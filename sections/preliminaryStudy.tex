% !TEX root = preliminaryStudy.tex

\subsection{Nettoyage des données}
Nous disposons de 1000 sessions d'individus qui ont effectué au moins une transaction, nous avons trouvé quelques individus qui apparaissent deux fois dans l'échantillon et nous voudrions composer les données avec des individus différents pour éviter de polluer  les données.

\subsection{Description des données}
Ci-dessous se trouve une description de chacune des variables de l'ensemble de données, à l'exception de la variable \textit{fullVisitorId} lequelle est un hash composé de 16 caractères alphanumériques pour identifier le visiteur.



\begin{center}
\begin{longtable}{ |p{3cm}||p{3cm}|p{2cm}|p{5cm}|  }
 \hline
 \multicolumn{4}{|c|}{Variables} \\
 \hline
 Nom & Type &Intervalle&Définition\\
 \hline
 hits   & Quantitative discrète  &[9,800]&  Les hits sont les interactions des utilisateurs sur le site web.\\
New visit &   Binaire   & [0, 1]   & C'est la première visite sur le site.\\
Page views&   Quantitative discrète  & [1, 137]   &Est une page chargée (ou rechargée) dans un navigateur.\\
Total transaction revenue & Quantitative continue  &[8.950, 871.380]&  Le revenu associé à la transaction. Cette valeur peut inclure les frais d'expédition, les taxes ou d'autres ajustements du revenu.\\
Transactions    &Quantitative discrète  & [1, 6] &  Nombre de transactions du client dans la session.\\
Time on site&   Quantitative continue  & [68, 8558]& Temps sur le site avant d'effectuer la transaction \\
 Browser name& Qualitative nominal  & -   & Nom du navigateur web.\\
 Browser width& Quantitative continue  & [320, 2560]& La largeur de la fenêtre du navigateur \\
 Browser height& Quantitative continue  & [280, 1356]& La hauteur de la fenêtre du navigateur\\
 Device category& Qualitative nominal   & -&Le dispositif par lequel l'utilisateur a effectué la transaction.\\
 Operating system& Qualitative nominal  & -&Le système d'exploitation par lequel l'utilisateur a effectué la transaction\\
 Language& Qualitative nominal  & -& La langue du navigateur au moment de la transaction.\\
 Country& Qualitative nominal   & -& Le pays depuis où l'utilisateur a effectué la transaction.\\
 Payment method& Qualitative nominal   & -&  Représente  ce que l'utilisateur a utilisé pour payer la transaction.\\
 Item count& Quantitative discrète  & [1, 18]& Un item représente un produit individuel acheté dans le cadre de la transaction. \\
 \hline
\end{longtable}
\end{center}

\subsection{Description des variables}
UnivariePour commencer, ce rapport comportera une analyse univariée de chacune des variables quantitatives et qualitatives qui font partie de le jeu de données, en suite il passera a l'analyse bivariée pour vérifier des liens entre les variables quantitatives mesurées par la corrélation linéaire.




