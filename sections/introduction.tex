% !TEX root = introduction.tex

(À REVOIR) Comprendre comment se comporte un client est souvent un défi, et cela devient encore plus difficile lorsqu'il s'agit du marché numérique, le marché numérique est en pleine évolution et les clients changent constamment de comportement. Les entreprises qui réussissent à tirer parti de l'analyse de données numériques récoltées sur les clients ont une longueur d'avance sur la concurrence. Les méthodes d’analyse de données sont un moyen important pour nous aider a trouver des réponses que nous permettent de analyser cette comportement. Grâce a les techniques d’analyse des données il est possible d'analyser, par exemple, si le comportement d'achat des utilisateurs varie-t-il en fonction de l'appareil qu'ils utilisent ou découvrir combien des interactions sont nécessaires avant d'une transaction. Cette information est précieuse et dans ce rapport on va mettre en application ces techniques pour décrire un jeu des données que contient 1000 individus  qui ont effectué au moins une transaction sur une site e-commerce français.

Pour commencer, ce rapport comportera une analyse univariée de chacune des variables quantitatives et qualitatives qui font partie de le jeu de données, en suite il passera a l'analyse bivariée pour vérifier des liens entre les variables quantitatives mesurées par la corrélation linéaire. En suite nous allons a étudier la forme du nuage des individus avec l'analyse de composants principaux qui va permettre d'interpréter le plan des données pour en suite appliquer l'algorithme K-means pour apprécier la proximité entre les individus et la variété de profils.