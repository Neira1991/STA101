Au vu de l'analyse bivarié, nous remarquons variables que s'approchant.


les variables actives étant à la fois quantitatives et qualitatives, le choix d'une AFDM pour une analyse factorielle s'imposera.

Notons que nous avions précédemment repéré de potentielles liaisons non-linéaires entre variables quantitatives. Malheureusement, l'AFDM (comme l'ACP) n'est pas une méthode très performante pour mettre en évidence des liaisons non-linéaires entre variables quantitatives. Pour corriger cela, nous avons choisi d'opérer les transformations log sur les variables relatives à la pollution. Cette transformation est assez courante pour des données de comptage (ici on compte des particules de polluant). La visualisation de la matrice représentant les coefficients de Pearson semble indiquer que cette transformation a du sens. En effet, on peut constater que les coefficients associés aux variables transformées sont globalement plus élevé que ceux portant sur les variables non-transformées.

browser heigth et width, hits y las que componen activity, operatingSystem, 

las variables activas

itemCount
activity
totalTransactionRevenue
sizeArea

country
language
browser name
paymentMethod
deviceCategory