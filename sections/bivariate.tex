% !TEX root = bivariate.tex
Une matrice de corrélation a été réalisée entre les variables quantitatives afin de pouvoir détecter les variables ayant une liaison. A partir de les resulta, on va créer une variable supplémentaire que s'appelle activité, pour pouvoir recapituler les info de Hits, pageviews et time on site y otra variable de aria comninando con heigth y weith

\begin{table}[ht]
\begin{adjustbox}{width=1\textwidth}
\small
\begin{tabular}{|l|c|c|c|c|c|c|c|}
\hline
                        & \multicolumn{1}{l|}{Hits} & \multicolumn{1}{l|}{pageviews} & \multicolumn{1}{l|}{totalTransactionRevenue} & \multicolumn{1}{l|}{timeOnSite} & \multicolumn{1}{l|}{browser\_width} & \multicolumn{1}{l|}{browser\_height} & \multicolumn{1}{l|}{itemCount} \\ \hline
hits                    & 1                         & \cellcolor{green!15} 0.74                           & 0.65                                         & 0.69                            & -0.04                               & -0.18                                & 0.56                           \\ \hline
pageviews               & \cellcolor{green!15}0.74                      & 1                              & 0.44                                         & \cellcolor{green!15}0.82                            & 0.07                                & -0.11                                & 0.45                           \\ \hline
totalTransactionRevenue & 0.65                      & 0.44                           & 1                                            & 0.44                            & 0.03                                & -0.06                                & 0.42                           \\ \hline
timeOnSite              & 0.69                      &\cellcolor{green!15} 0.82                           & 0.44                                         & 1                               & 0.02                                & -0.17                                & 0.47                           \\ \hline
browser\_width          & -0.04                     & 0.07                           & 0.03                                         & 0.02                            & 1                                   & \cellcolor{green!15}0.77                                 & -0.02                          \\ \hline
browser\_height         & -0.18                     & -0.11                          & -0.06                                        & -0.17                           &\cellcolor{green!15} 0.77                                & 1                                    & -0.19                          \\ \hline
itemCount               & 0.56                      & 0.45                           & 0.42                                         & 0.47                            & -0.02                               & -0.19                                & 1                              \\ \hline
\end{tabular}
\end{adjustbox}
\end{table}
 
Nous créons une table de contigence pour voir sur quel genre d'appareil se trouvent les transactions présentées par navigateur. nous pouvons observer que Chrome Mobile est un navigateur pour téléphones mobiles et tablets et chrome un navigateur juste pour desktop, en revanche,  Safari est un navigateur transversal aux dispositifs.
\begin{table}[ht]
\begin{adjustbox}{width=1\textwidth}
\small
\begin{tabular}{|l|c|c|c|c|c|c|c|}
\hline
             & \multicolumn{1}{l|}{Chrome} & \multicolumn{1}{l|}{Chrome Mobile} & \multicolumn{1}{l|}{Edge} & \multicolumn{1}{l|}{Firefox} & \multicolumn{1}{l|}{Internet Explorer} & \multicolumn{1}{l|}{Safari} & \multicolumn{1}{l|}{Samsung Browser} \\ \hline
Desktop      & 353                         & 0                                  & 29                        & 61                           & 8                                      & 58                          & 0                                    \\ \hline
Mobile Phone & 0                           & 178                                & 0                         & 1                            & 0                                      & 134                         & 23                                   \\ \hline
Tablet       & 0                           & 9                                  & 1                         & 0                            & 0                                      & 25                          & 5                                    \\ \hline
\end{tabular}
\end{adjustbox}
\end{table}


Nous avons décidé de supprimer les transactions pour les navigateurs Iron, MIUI Browser et Opera, car il n'y avait que 1 ou 2 transactions pour ce navigateur dans le chantillon, et les résultats pour ce navigateur ne seront pas représentatifs.



\begin{table}[]
\begin{tabular}{|l|c|c|c|}
\hline
        & \multicolumn{1}{l|}{Iron} & \multicolumn{1}{l|}{MIUI} & \multicolumn{1}{l|}{Opera} \\ \hline
Desktop & 1                         & 0                         & 2                          \\ \hline
Mobile  & 0                         & 1                         & 0                          \\ \hline
Tablet  & 0                         & 0                         & 0                          \\ \hline
\end{tabular}
\end{table}

