% !TEX root = bivariate.tex
Une matrice de corrélation a été réalisée entre les variables quantitatives afin de détecter les  variables ayant une liaison. À partir des résultats, nous avons décidé de créer une variable  supplémentaire nommée \emph{\textbf{Activity}}, qui englobe les informations des variables  \emph{Hits}, \emph{PageViews} et \emph{TimeOnSite}. Nous remarquons également que les  variables \emph{browser\_width} et \emph{browser\_height} sont corrélés, nous pouvons donc  créer une variable \emph{\textbf{Area}} à partir de ces deux variables, elle représentera la surface  du dispositif de l'utilisateur. 

\begin{table}[ht] 
\begin{adjustbox}{width=1\textwidth} 
\small 
\begin{tabular}{|l|c|c|c|c|c|c|c|} 
\hline 
& \multicolumn{1}{l|}{Hits} & \multicolumn{1}{l|}{pageviews} &  
\multicolumn{1}{l|}{totalTransactionRevenue} & \multicolumn{1}{l|}{timeOnSite} & 
\multicolumn{1}{l|}{browser\_width} & \multicolumn{1}{l|}{browser\_height} &  \multicolumn{1}{l|}{itemCount} \\ \hline 
hits & 1 & \cellcolor{green!15} 0.74 & 0.65 & 0.69 & -0.04 & -0.18 & 0.56 \\ \hline pageviews & \cellcolor{green!15}0.74 & 1 & 0.44 & \cellcolor{green!15}0.82 & 0.07 & -0.11 & 0.45 \\ \hline 
totalTransactionRevenue & 0.65 & 0.44 & 1 & 0.44 & 0.03 & -0.06 & 0.42 \\ \hline timeOnSite & 0.69 &\cellcolor{green!15} 0.82 & 0.44 & 1 & 0.02 & -0.17 & 0.47 \\ \hline browser\_width & -0.04 & 0.07 & 0.03 & 0.02 & 1 & \cellcolor{green!15}0.77 & -0.02 \\ \hline browser\_height & -0.18 & -0.11 & -0.06 & -0.17 &\cellcolor{green!15} 0.77 & 1 & -0.19 \\ \hline itemCount & 0.56 & 0.45 & 0.42 & 0.47 & -0.02 & -0.19 & 1 \\ \hline 
\end{tabular} 
\end{adjustbox} 
\end{table} 
Nous créons une table de contingence pour voir sur quel type d'appareil se trouvent les transactions  présentées par le navigateur. Nous observons que Chrome Mobile est un navigateur pour  téléphones mobiles et tablets et chrome un navigateur juste pour desktop, en revanche, Safari est  un navigateur transversal aux dispositifs. 
\begin{table}[ht] 
\begin{adjustbox}{width=1\textwidth} 
\small 
\begin{tabular}{|l|c|c|c|c|c|c|c|} 
\hline 
& \multicolumn{1}{l|}{Chrome} & \multicolumn{1}{l|}{Chrome Mobile} & \multicolumn{1}{l|}{Edge}  & \multicolumn{1}{l|}{Firefox} & \multicolumn{1}{l|}{Internet Explorer} &  
\multicolumn{1}{l|}{Safari} & \multicolumn{1}{l|}{Samsung Browser} \\ \hline Desktop & 353 & 0 & 29 & 61 & 8 & 58 & 0 \\ \hline 
Mobile Phone & 0 & 178 & 0 & 1 & 0 & 134 & 23 \\ \hline 
Tablet & 0 & 9 & 1 & 0 & 0 & 25 & 5 \\ \hline 
\end{tabular} 
\end{adjustbox} 
\end{table} 
Nous avons décidé de supprimer les transactions pour les navigateurs Iron, MIUI Browser et Opera,  car il n'y a que 1 ou 2 transactions pour ce navigateur dans l’échantillon. Les résultats pour ce  navigateur ne sont pas représentatifs. 
\begin{table}[ht] 
\begin{adjustbox}{width=0.4\textwidth} 
\begin{tabular}{|l|c|c|c|} 
\hline 
& \multicolumn{1}{l|}{Iron} & \multicolumn{1}{l|}{MIUI} & \multicolumn{1}{l|}{Opera} \\ \hline Desktop & 1 & 0 & 2 \\ \hline 
Mobile & 0 & 1 & 0 \\ \hline
Tablet & 0 & 0 & 0 \\ \hline 
\end{tabular} 
\end{adjustbox} 
\end{table} 
\begin{table}[!hbtp] 
\caption{ \emph{deviceCategory} et le \emph{operatingSys} en pourcentages}  \begin{adjustbox}{width=0.6\textwidth} 
\begin{tabular}{|l|c|c|c|} 
\hline 
& \multicolumn{1}{l|}{Desktop} & \multicolumn{1}{l|}{Mobile} & \multicolumn{1}{l|}{Tablet} \\ \hline 
Android & 0 & 93.9 & 6.1 \\ \hline 
iOS & 0 & 83.5 & 16.5 \\ \hline 
Linux & 100 & 0 & 0 \\ \hline 
OS X & 100 & 0 & 0 \\ \hline 
Windows 10 & 100 & 0 & 0 \\ \hline 
Windows 7 & 100 & 0 & 0 \\ \hline 
Windows 8 & 100 & 0 & 0 \\ \hline 
Windows 8.1 & 100 & 0 & 0 \\ \hline 
Windows Vista & 100 & 0 & 0 \\ \hline 
All & 57.6 & 37.9 & 4.5 \\ \hline 
\end{tabular} 
\end{adjustbox} 
\label{table:deviceCategoryoperatingSys} 
\end{table} 
Nous voulons croiser la catégorie \emph{deviceCategory} et le \emph{operatingSys}, on hypothétise  qu'elles sont liées. Dans le tableau ~\ref{table:deviceCategoryoperatingSys} on représente les  effectifs dans le tableau en pourcentages pour mieux interpréter. 
\begin{table}[!hbtp] 
\caption{statistique de test}  
\begin{center} 
\begin{tabular}{lcccc} 
\hline\hline 
\multicolumn{1}{c}{Desktop}&\multicolumn{1}{c}{Mobile}&\multicolumn{1}{c}{Tablet}&\multicolum n{1}{c}{Resultat}\tabularnewline 
\multicolumn{1}{c}{{\scriptsize $N=512$}}&\multicolumn{1}{c}{{\scriptsize  
$N=337$}}&\multicolumn{1}{c}{{\scriptsize $N=40$}}&\tabularnewline 
\hline 
57\%&38\%&5\%&$ \chi^{2}_{2}=913.73,~ P=< 2.2e-16 $\tabularnewline 
\hline 
\end{tabular}
\end{center} 
\label{table:testChiDeux} 
\end{table} 
Comme nous travaillons avec un échantillon et non une population entière, nous pouvons compléter  ce tableau croisé par un test d'indépendance de \emph{khi-deux}. Cela nous permet de tester, et  éventuellement de rejeter, l'hypothèse d'indépendance des lignes et des colonnes du tableau, c'est à-dire l'hypothèse selon laquelle les écarts observés par rapport à l'indépendance sont uniquement  dus à un biais d'échantillonnage. Dans le tableau ~\ref{table:testChiDeux} la statistique de test est  plus élevée que la valeur \emph{khi-deux}, donc on ne peut pas maintenir l'hypothèse selon laquelle  le type de dispositif n'a aucun lien avec le système d'exploitation. 
